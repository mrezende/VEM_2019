%%%%%%%%%%%%%%%%%%%%%%%%%%%%%%%%%%%%%%%%%%%%%%%%%%%%%%%%%%%%%%%%%%%%%%
% How to use writeLaTeX: 
%
% You edit the source code here on the left, and the preview on the
% right shows you the result within a few seconds.
%
% Bookmark this page and share the URL with your co-authors. They can
% edit at the same time!
%
% You can upload figures, bibliographies, custom classes and
% styles using the files menu.
%
%%%%%%%%%%%%%%%%%%%%%%%%%%%%%%%%%%%%%%%%%%%%%%%%%%%%%%%%%%%%%%%%%%%%%%

\documentclass[12pt]{article}

\usepackage{sbc-template}

\usepackage{graphicx,url}

%\usepackage[brazil]{babel}   
\usepackage[utf8]{inputenc}  
\renewcommand{\refname}{Referências}

     
\sloppy

\title{Estudo preliminar: Uso da arquitetura deep learning de seleção de respostas no problema de recuperação de código-fonte}

\author{Marcelo de Rezende Martins\inst{1}, Marco Aurélio Gerosa\inst{2}}


\address{Instituto de Pesquisas Tecnológicas
  (IPT)\\
  São Paulo -- SP -- Brazil
\nextinstitute
  Northern Arizona University\\
  Flagstaff, AZ, US
  \email{rezende.martins@gmail.com, Marco.Gerosa@nau.edu}
}

\begin{document} 

\maketitle

\begin{abstract}
  This meta-paper describes the style to be used in articles and short papers
  for SBC conferences. For papers in English, you should add just an abstract
  while for the papers in Portuguese, we also ask for an abstract in
  Portuguese (``resumo''). In both cases, abstracts should not have more than
  10 lines and must be in the first page of the paper.
\end{abstract}
     
\begin{resumo} 
  O problema de recuperação de código-fonte consiste em dado uma questão e um conjunto de trechos de código-fonte, recuperar o trecho de código-fonte que resolva a questão . Este artigo busca apresentar uma nova abordagem para este problema, fazendo uma aproximação com o problema de seleção de respostas da área de processamento de linguagens naturais (\emph{NLP}). Sob esta perspectiva, pretendemos avaliar, inicialmente, o desempenho da arquitetura de rede neural recorrente bidirecional (bi-LSTM) com uma estrutura de rede neural convolucional (CNN) no problema de recuperação de código-fonte. 
\end{resumo}


\section{Introdução}

Segundo \cite{Allamanis-bimodal-source-code-natural-language:2015}, recuperação de código-fonte ou \textit{code retrieval} é um problema de recuperar informação, onde dado uma questão ou uma descrição em linguagem natural e um conjunto de possíveis trechos de código-fonte, o objetivo é recuperar o trecho de código-fonte que solucione a questão ou seja mais relevante de acordo com a descrição. Já segundo \cite{lai-etal-2018-review},
dado uma questão e um conjunto de possíveis respostas, seleção de respostas ou \textit{answer selection} busca identificar qual resposta consegue responder a 
pergunta corretamente. Pela definição, é possível verificar as similaridades dos problemas. \cite{iyer-etal-2016-summarizing} em seu trabalho, utilizou LSTM com o mecanismo de atenção para solucionar
o problema de \textit{code retrieval}. Tanto LSTM quanto o mecanismo de atenção é comumente utilizado nos problemas de tradução e geração de texto. Para \cite{Knuth:1984:LP}, 
o código-fonte é uma forma de comunicação:

\textit{"Let us change our traditional attitude to the construction of programs: Instead of imagining that our
main task is to instruct a computer what to do, let us
concentrate rather on explaining to human beings what
we want a computer to do."}

E \cite{Allamanis:2018:SML} apresenta a hipótese da naturalidade do código-fonte:

\textit{"\textbf{The naturalness hypothesis}. Software is a form of human communication; soft-
ware corpora have similar statistical properties to natural language corpora; and these
properties can be exploited to build better software engineering tools."}

Considerando o código-fonte uma forma de comunicação, assim elencado por \cite{Knuth:1984:LP} e reforçado por \cite{Allamanis:2018:SML} em sua hipótese, uma possível abordagem para o problema 
do \textit{code retrieval}, é analisá-la sob a perspectiva do \textit{answer selection}. Porém, é necessário levar em consideração as peculiaridades do código-fonte ao abordar o problema sob esta perspectiva. 

Este presente artigo busca apresentar os resultados preliminares de um estudo sobre o uso de arquitetura deep learning de answer selection no problema do code retrieval.
Inicialmente, utilizaremos a arquitetura bi-LSTM com CNN proposta por \cite{tan-lstm-qa}. Além de propor uma nova arquitetura para este problema, iremos avaliar o modelo 
utilizando os dados de entrada da base StaQC, criada por \cite{Yao-staqc:2018}. \cite{Yao-staqc:2018} coletou milhares de pares de perguntas e trechos de código-fonte do StackOverFlow e a 
disponibilizou publicamente. \cite{Yao-staqc:2018} avaliou a base utilizando a arquitetura proposta por \cite{iyer-etal-2016-summarizing}, que combina uma rede neural LSTM com o mecanismo 
de atenção (attention). Esta arquitetura utilizada por \cite{iyer-etal-2016-summarizing} é comumente utilizada na geração de textos, que foi a proposta do trabalho de \cite{iyer-etal-2016-summarizing}. Além de criar um modelo
para recuperar o trecho de código-fonte a partir de uma descrição em linguagem natural, o modelo proposto por \cite{iyer-etal-2016-summarizing} foi utilizada para fazer o inverso, gerar uma descrição
a partir de um trecho de código-fonte. A nossa proposta inicial é apenas recuperar o trecho de código-fonte a partir de uma descrição.

Na seção método é apresentado a arquitetura utilizada e as adaptações necessárias feitas. Na seção resultados iniciais discute-se os resultados e os ajustes nos 
parâmetros para evitar overfitting. Próximos passos e trabalhos futuros é apresentado na seção conclusão.


\section{Método} \label{sec:metodo}

Conforme mencionado anteriormente, iremos abordar o problema do \textit{code retrieval} sob a perspectiva do \textit{answer selection}. Utilizaremos inicialmente
a arquitetura proposta por Tan et al., que utiliza a função de classificação \textit{pairwise} e uma arquitetura siamesa \cite{lai-etal-2018-review}. De acordo com 
\cite{lai-etal-2018-review}, o problema de \textit{answer selection} pode ser analisado sob duas formas: forma de aprendizado e arquitetura.

\subsection{Forma de aprendizado}

O problema de \textit{answer selection} consiste em encontrar a resposta mais relevante dado uma questão. Ele pode ser abordado como uma problema de classificação, onde
o objetivo é classificar com uma pontuação melhor as respostas mais relevantes de acordo com a questão.

Tan et al. utilizou o método \textit{pairwise}, no qual ele classifica as respostas corretas com uma pontuação maior que a pontuação das incorretas. Dado uma questão,
o método utiliza pares de candidatos de repostas e aprende a classificar qual resposta é mais relevante para a questão. Por exemplo, os dados de entrada de treinamento
são triplas $(q_{i}, c_{i}^{+}, c_{i}^{-})$, onde $q_{i}$ é uma questão, $c_{i}^{+}$ é uma resposta correta, $c_{i}^{-}$ é uma resposta incorreta obtida a partir da amostra completa dos dados. 

E a função hinge loss é definida como:

\begin{equation}
L = max(0, m - h_{\theta}(q_{i}, c_{i}^{+}) + h_{\theta}(q_{i}, c_{i}^{-}))   
\end{equation}


Onde $m$ é a margem. Se $h_{\theta}(q_{i}, c_{i}^{+}) - h_{\theta}(q_{i}, c_{i}^{-}) < m$ então $L$ é positivo. Quando esta condição é satisfeita, a implicação é que o sistema classifica a resposta correta abaixo da resposta incorreta, ou a questão correta pontua um pouco acima da resposta incorreta. Por outro lado, se a resposta correta tem uma pontuação maior que a incorreta por uma margem acima ou igual a $m$ (i.e., $h_{\theta}(q_{i}, c_{i}^{+}) - h_{\theta}(q_{i}, c_{i}^{-}) \geq m$, a função hinge loss é igual a zero. Em resumo, a função loss incentiva a reposta correta a ter uma pontuação maior que a incorreta por uma certa margem \cite{lai-etal-2018-review}.

\cite{tan-lstm-qa} propõe o uso da função de similaridade \textit{cosine}, na função $h_{\theta}$, pois esta ignora a magnitude e foca somente na orientação.
Isto ajuda a identificar um documento muito longo com o mesmo tema que um documento curto no problema de recuperação de informação em NLP, por exemplo. Pois a função não se preocupa
com a magnitude ou o tamanho do documento em si. 

Esta função de similaridade é comumente utilizada em problemas de \textit{answer selection} \cite{feng-answer-selection-2015}. Normalmente as questões são mais curtas que as respostas. No nosso caso, as questões tem em média 8 tokens, enquanto os trechos de código-fonte tem 48 tokens.

\subsection{Arquitetura}

\cite{tan-lstm-qa} propôs uma arquitetura QA-LSTM que utiliza uma rede neural recorrente bidirecional, mais especificamente uma rede LSTM (biLSTM) \cite{hochreiter-Schmidhuber-lstm-1997} e uma pooling layer para construir a representação dos vetores de entrada indepdentemente. Ao final, o modelo utiliza a função de similaridade \textit{cosine} para calcular a distância
entre as representações. 

A arquitetura de referência que iremos utilizar acrescenta uma estrutura de rede convolucional na saída da rede bi-LSTM. Os vetores de saída da rede bi-LSTM tornam-se vetores de entrada na na estrutura CNN. O CNN auxilia na síntese da representação dos vetores.

\section{Experimento}\label{sec:experimento}

Para avaliar a arquitetura proposta por \cite{tan-lstm-qa} composta por uma rede bi-LSTM com CNN no problema do \textit{code retrieval}, utilizamos os dados disponibilizados por \cite{Yao-staqc:2018}. Em seu trabalho \cite{Yao-staqc:2018} coletou mais de \textbf{147 mil} pares de perguntas e trechos de código-fonte em Python e aproximadamente \textbf{119 mil} pares de perguntas e trechos de código-fonte em SQL. 

Inicialmente, utilizaremos uma amostra de 62252 pares $(q_{i}, c_{i}^{+})$ em Python. Onde $q_{i}$ é o título de uma questão no site StackOverFlow\footnote{Site: https://www.stackoverflow.com/} e $c_{i}^{+}$ corresponde a um trecho de código-fonte apontado como solução para a questão $q_{i}$. Destes 62252 pares, excluímos 2169 pares que foram anotados manualmente por 4 estudantes de graduação. Segundo \cite{Yao-staqc:2018}, os trechos de código-fonte que são provável solução foram anotados com a anotação \emph{"1"}, caso contrário \emph{"0"}. Todos os pares $(q_{i}, c_{i})$ foram revisados por pelo menos dois alunos e somente é considerado solução para a questão, o trecho de código-fonte anotado com a anotação \emph{"1"} por todos os estudantes.


\begin{table}[h]
\centering
\begin{tabular}{ |p{3cm}|p{3cm}|  }
 \hline
 \textbf{Amostras} & \textbf{Quantidade de $(q_{i}, c_{i}^{+})$}\\
 \hline
 Treinamento & $60083$\\
 \hline
 DEV & $1085$ \\
 \hline
 EVAL & $1084$\\
 \hline
 \textbf{Total} & $62252$\\
 \hline
\end{tabular}
\caption{Divisão das amostras conforme os critérios adotados por \cite{iyer-etal-2016-summarizing}}
\label{table:divisao-amostras}
\end{table}

Conforme a tabela~\ref{table:divisao-amostras}, utilizamos 60083 pares $(q_{i}, c_{i}^{+})$ para treinamento. Estes pares foram anotados automaticamente por um framework proposto por \cite{Yao-staqc:2018}. Este framework é composto por uma rede neural recorrente que foi treinada e avaliada nos dados anotados manualmente pelos estudantes. Ao final, o modelo obtido a partir do treinamento, classificou automaticamente 60083 pares de $(q_{i}, c_{i}^{+})$ em Python, por exemplo.

No nosso caso, utilizaremos estes dados anotados automaticamente para treinar o nosso modelo. E para avaliar inicialmente o modelo do \cite{tan-lstm-qa}, utilizaremos os 2169 pares $(q_{i}, c_{i}^{+})$ anotados manualmente.

Para minimizar o viés, iremos adotar o procedimento proposto por \cite{iyer-etal-2016-summarizing} na avaliação do modelo. O modelo obtido durante o treinamento será avaliado a cada época na amostra \emph{DEV}. Para cada par $(q_{i}, c_{i}^{+})$ na amostra \emph{DEV}, selecionaremos outros 49 distratores $c^{'}$ da amostra de treinamento, onde $c^{'} \neq c_{i}^{+}$. Estes 50 pares $(c_{i}, q_{i})$ serão classificados de acordo com a função $h_{\theta}$ de similaridade. Posteriormente, o MRR de cada questão $q_{i}$ da amostra \emph{DEV} será calculada. Ao final calculamos a média da MRR. O modelo de treinamento que obtiver a maior média MRR na amostra \emph{DEV} será escolhida. 

A avaliação final é feita utilizando o modelo obtido no treinamento. O procedimento é o mesmo, porém o modelo é avaliado na amostra \emph{EVAL}. 


\subsection{Configuração}

Os dados foram representados como sequência de tokens. Utilizamos uma representação distribuída word2vec \cite{mikolov-word2vec-2013}. Diferente do \textit{answer selection} proposto por \cite{tan-lstm-qa} no qual ele cria apenas uma representação distribuída para a amostra inteira, nós geramos a representação distribuída para as questões e outra para os trechos do código-fonte.


\subsection{Resultados preliminares}\label{sec:resultados-preliminares}

Os resultados do experimento podem ser visualizados na tabela~\ref{table:resultados-preliminares}. Os resultados correspondem a média do MRR após 20 rodadas de execução utilizando o melhor modelo obtido a partir do treinamento. E o modelo foi avaliado utilizado os dados EVAL. De acordo com a tabela~\ref{table:resultados-preliminares}, \textbf{bi-LSTM-CNN} obteve o melhor desempenho. Porém, CNN obteve um resultado muito próximo e o tempo de treinamento é muito menor. O tempo total de treinamento da rede bi-LSTM com CNN levou em torno de 48 minutos, utilizando uma GPU Tesla K80. Enquanto a arquitetura CNN levou em torno de 6s. 

Em todos os modelos, utilizamos uma camada de maxpool e a função de similaridade \textit{cosine}. Utilizamos o valor de margem $0,009$ para a função \textit{hinge loss}, valor proposto por \cite{feng-answer-selection-2015}.


\begin{table}[h]
\centering
\begin{tabular}{ |p{3cm}|p{3cm}|  }
 \hline
 \textbf{Modelos} & \textbf{MRR}\\
 \hline
 Embedding & $0,52 \pm 0,01$\\
 \hline
 CNN & $0,58 \pm 0,01 $ \\
 \hline
 \textbf{bi-LSTM-CNN} & $\textbf{0,60} \pm \textbf{0,02}$\\
 \hline
\end{tabular}
\caption{Resultados preliminares dos modelos QA de \cite{tan-lstm-qa} no problema de \textit{code retrieval}}
\label{table:resultados-preliminares}
\end{table}

\subsection{Ameaças à validade}

Conforme exposto na seção~\ref{sec:experimento}, os dados utilizados no treinamento para obter o modelo final foram coletados automaticamente por \cite{Yao-staqc:2018}. \cite{Yao-staqc:2018} criou um modelo composto por uma rede neural recorrente para obter os pares de questões e trechos de código-fonte automaticamente. E para treinar este modelo, foi utilizado os dados anotados manualmente. Estes dados anotados manualmente são os mesmos utilizados na avaliação do nosso modelo. 

Para minimizar o viés, utilizamos o mesmo procedimento de avaliação proposto por \cite{iyer-etal-2016-summarizing}. Para cada par de $(q_{i}, c_{i}^{+})$, selecionamos aleatoriamente 49 distratores $c^{'}$ da amostra de treinamento, onde $c^{'} \neq c_{i}$. 

\section{Conclusões}

Neste trabalho, propusemos uma abordagem diferente para o problema do \textit{code retrieval}. Nosso intuito foi abordar o problema do \textit{code retrieval} sob a perspectiva do problema \textit{answer selection}, já conhecido em NLP. E dado o bom desempenho dos modelos deep learning no contexto do \textit{answer selection}, utilizamos o modelo proposto por \cite{tan-lstm-qa}.

Conforme a seção~\ref{sec:resultados-preliminares}, o modelo proposto por \cite{tan-lstm-qa} apresentou um bom desempenho no problema do \textit{code retrieval}. 



\bibliographystyle{sbc}
\bibliography{sbc-template}


\end{document}